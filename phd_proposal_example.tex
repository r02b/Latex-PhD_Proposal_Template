% %%%%%%%%%%%%%%%%%%%%%%%%%%%%%%%%%%%%%%%%%%%%%%%%%
%   PhD Proposal Latex Template
%   By Noga H. Rotman
%
%   This file will show you how to use this package.
%
% %%%%%%%%%%%%%%%%%%%%%%%%%%%%%%%%%%%%%%%%%%%%%%%%%

% \documentclass{phd_proposal}
\documentclass[widemargins, newpage]{phd_proposal}
% \documentclass[draft]{phd_proposal}

% \documentclass[removefigures]{phd_proposal}


\usepackage{lipsum}
\usepackage{hyperref}


%%%%% Info for the reseach proposal %%%%%
% \title{PhD Proposal template}
\title{A really cool and long title for your awesome research proposal}
\author{You!}
\supervisor{Your Supervisor}
\institution{Some University Name}
\school{School of Nonsense}
\logo{png} % Optional. The logo file should be named 'logo' and be located in the 'figures' folder. The value is the format of the file (could be JPG, PNG,...)
\logowidth{0.2} % Optional, only works if the 'logo' variable is given. Width of the logo is calculated as a ration to the text width. Default is 0.1.


%%%%% Document %%%%%
\begin{document}

\maketitle

\begin{abstract}
\lipsum[1]
\end{abstract}

% \newpage
% \input{introduction}

\section{Introduction}
This template~\cite{github-repo} should be easy enough to use out of the box, but there are some cool features you might want to play around with.
It is given under the MIT license~\cite{mit-license}.

\section{Usage}

See the README.MF file




\bibliographystyle{abbrv}
\begin{small}
\bibliography{references.bib}
\end{small}
\end{document}